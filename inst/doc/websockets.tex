% \VignetteIndexEntry{websockets Manual}
% \VignetteDepends{websockets}
% \VignettePackage{websockets}
\documentclass[12pt]{article}
\usepackage{amsmath}
\usepackage[pdftex]{graphicx}
\usepackage{color}
\usepackage{xspace}
\usepackage{fancyvrb}
\usepackage{fancyhdr}
\usepackage[
     colorlinks=true,
     linkcolor=blue,
     citecolor=blue,
     urlcolor=blue]
     {hyperref}
\usepackage{lscape}
\usepackage{Sweave}
\usepackage{tabularx}
\usepackage{listings}
\usepackage{mdwlist}

\usepackage[T1]{fontenc}
\usepackage{lmodern}
%\renewcommand{\ttdefault}{lmodern}

%%%%%%%%%%%%%%%%%%%%%%%%%%%%%%%%%%%%%%%%%%%%%%%%%%%%%%%%%%%%%%%%%%

% define new colors for use
\definecolor{darkgreen}{rgb}{0,0.6,0}
\definecolor{darkred}{rgb}{0.6,0.0,0}
\definecolor{lightbrown}{rgb}{1,0.9,0.8}
\definecolor{brown}{rgb}{0.6,0.3,0.3}
\definecolor{darkblue}{rgb}{0,0,0.8}
\definecolor{darkmagenta}{rgb}{0.5,0,0.5}
\definecolor{lgray}{rgb}{0.95,0.95,0.95}

%%%%%%%%%%%%%%%%%%%%%%%%%%%%%%%%%%%%%%%%%%%%%%%%%%%%%%%%%%%%%%%%%%

\newcommand{\bld}[1]{\mbox{\boldmath $#1$}}
\newcommand{\shell}[1]{\mbox{$#1$}}
\renewcommand{\vec}[1]{\mbox{\bf {#1}}}
\newcommand{\ReallySmallSpacing}{\renewcommand{\baselinestretch}{.6}\Large\normalsize}
\newcommand{\SmallSpacing}{\renewcommand{\baselinestretch}{1.1}\Large\normalsize}
\def\tm{\leavevmode\hbox{$\rm {}^{TM}$}}


\setlength{\oddsidemargin}{-.25 truein}
\setlength{\evensidemargin}{0truein}
\setlength{\topmargin}{-0.2truein}
\setlength{\textwidth}{7 truein}
\setlength{\textheight}{8.5 truein}
\setlength{\parindent}{0.20truein}
\setlength{\parskip}{0.10truein}

%%%%%%%%%%%%%%%%%%%%%%%%%%%%%%%%%%%%%%%%%%%%%%%%%%%%%%%%%%%%%%%%%%
\pagestyle{fancy}
\lhead{}
\chead{The {\tt websockets} Package}
\rhead{}
\lfoot{}
\cfoot{}
\rfoot{\thepage}
\renewcommand{\headrulewidth}{1pt}
\renewcommand{\footrulewidth}{1pt}
%%%%%%%%%%%%%%%%%%%%%%%%%%%%%%%%%%%%%%%%%%%%%%%%%%%%%%%%%%%%%%%%%%
\title{The {\tt websockets} Package}
\author{Bryan W. Lewis \\ 
blewis@illposed.net}

\begin{document}

\lstset{
  frame=single,
  columns=fixed,
  basicstyle=\ttfamily,
  backgroundcolor=\color{lgray},
  basewidth={0.50em, 0.45em},
  breaklines=true,
}

\maketitle

\thispagestyle{empty}

\section{Introduction}

HTML 5 websockets define an efficient socket-like communication protocol for
the web. The {\tt websockets} package is a native websocket implementation for
R that supports most of the draft IETF protocols in use today by web browsers.
The {\tt websockets} package is especially well-suited to interaction between R
and web scripting languages like Javascript.  Multiple simultaneous websocket
server and client connections are supported.
The package has few  dependencies and is written mostly in R.
Packages are available for all major R platforms including
GNU/Linux, OS X, and Windows.

Websockets are a particularly simple way to expose R to the Web as a
service--they let Javascript and other scripts embedded in web pages directly
interact with R, bypassing traditional middleware layers like .NET, Java, and
web servers normally used for such interaction. In some cases, websockets
can be much more efficient than traditional Ajax schemes for interacting with
clients over web protocls. Websockets also simplify service scalability in
many cases.

The {\tt websockets} package provides three primary capabilities:
\begin{enumerate}
\item An websocket service.
\item An websocket client.
\item A basic HTTP service.
\end{enumerate}
This guide illustrates each capability with simple examples.

\section{Running an R {\tt websockets} server, step by step}

The {\tt websockets} package includes a server function that can initiate and
respond to websocket and HTTP events over a network connection
(websockets are an extension of standard HTTP).
All R/Websocket server applications share the following basic recpie:
\begin{enumerate}
\item Load the library.
\item Initialize a websocket server with {\tt create\_server}.
\item Set callback functions that will respond to desired events.
\item Service the server's socket interface with {\tt service}, often in an
      event loop.
\item Shutdown the server and delete the server environment when done.
\end{enumerate}
We outline the steps with examples below.

\subsection{Load the library}
\begin{lstlisting}
library("websockets")
\end{lstlisting}
The library depends on the {\tt caTools}, {\tt bitops} and a recent version of
the {\tt digest} package.  It suggests that the {\tt RJSONIO} library be
installed, as it is quite useful to have available when interacting with
Javascript.

\subsection{Initialize a websocket server with {\tt create\_server}}

The R/Websocket service is initialized by a call to the {\tt create\_server}
function. (The initialization method called {\tt createContext} from older
versions of the package is still supported.) The function takes two arguments,
a network port to listen on, and an optional function closure to service
standard HTTP requests (described in greater detail below).  The {\tt
create\_server} function returns an environment that stores data associated
with the newly created server.  ``Callback'' functions may be assigned that
respond to specific websocket and generic http events. Here is an example 
that creates a websocket server on the default port of 7681:
\begin{lstlisting}
server = create_server()
\end{lstlisting}
Multuple websocket servers may be defined, but they must use distinct ports.

The websocket server will respond directly to any websocket client request.
For convenience, the server may optionally also service basic, non-websocket
HTTP reuests.  For example, the basic package demo available from {\tt
demo('websockets')} serves clients the file {\tt basic.html} located in the
package installation path.  Additional examples are provided below. See the
\href{http://cran.r-project.org/web/packages/Rook/}{Rook} package for an
alternate comprehensive R web service.

\subsection{Set callback functions to respond to events}
Clients may connect to the websocket service immediately after the server is
initialized. The server may write data to or close client connections at any
time.  However, one must define functions to respond to incoming client events.

Each websocket server instance supports the following incoming events:
\begin{enumerate}
\item[] {\tt established}: Occurs when a websocket client connection is
successfully negotiated.
\item[] {\tt closed}: Occurs when a client websocket connection has been closed.
\item[] {\tt receive}: Occurs when data is received from a connection.
\end{enumerate}
R functions may be defined to handle some, all, or none of the above event
types. Such functions are termed ``callbacks.''

The {\tt set\_callback} function may be used to define a callback function in
the server environment returned by {\tt create\_server}. (It
simply assigns the functions in that environment.)

The {\tt receive} callback function 
must take precisely three parameters that are filled in by the library
with values corresponding to an event that invokes the callback.
The required parameters are:
{\tt DATA}: A vector of type raw that holds any incoming data associated
with the event. (It may be of length zero if the event does not have any
data to report.);
{\tt WS}: The websocket client associated with the event, represented
as an R list;
{\tt HEADER}: Header data returned by newer protocol versions, or
NULL for protocol version 00. The complete frame header is returned
as described in the IETF Draft
\href{ http://tools.ietf.org/html/draft-ietf-hybi-thewebsocketprotocol-17}
{http://tools.ietf.org/html/draft-ietf-hybi-thewebsocketprotocol-17}.

The {\tt closed} and {\tt receive} functions must each take one argument,
a {\tt WS} websocket client associated with the event, represented
as an R list.

The following example {\tt established} 
function sends a text message to each newly-established
connection:
\begin{lstlisting}
f = function(WS) {
  websocket_write("Hello there!", WS)
}
set_callback("established", f, server)
\end{lstlisting}

Here is an example {\tt receive} callback that receives data from a client
connection and simply echoes it back to the client:
\begin{lstlisting}
g = function(DATA, WS, ...) {
  websocket_write(DATA, WS)
}
setCallback("receive", g, server)
\end{lstlisting}

\subsection{Accept requests from web clients}
Javascript and other web script clients can very easily interact with the
R {\tt websockets} library directly from most browsers. The listing below
presents a very basic example client web page that includes Javascript
code to open a connection to a local websocket server running on port
7681. See the demo scripts in the
the package installation path for more complete examples.
\begin{lstlisting}
<html><body>
<script>
socket = new WebSocket("ws://localhost:7681", "chat");
try {
  socket.onmessage = function got_packet(msg) {
    document.getElementById("output").textContent = msg.data;
  } 
catch(ex) {document.getElementById("output").textContent = "Error: " + ex;}
</script>
<div id="output"> SOCKET DATA APPEARS HERE </div>
</body></html>
\end{lstlisting}
{\bf Note: The {\tt websockets} package presently ignores the
sub-protocol (``chat'' in the above example).}

\subsection{Service the socket interface with {\tt service}}

Incoming websocket events are queued. The {\tt service} function
processes events on a first-come, first-served basis.
The {\tt service} function processes each event by invoking the
appropriate callback function.
It returns after a configurable time out if there are no events to service.
Events may be processed indefinitely by evaluating the {\tt service} function
in a loop, for example:
\begin{lstlisting}
while(TRUE)
{
  service(server)
}
\end{lstlisting}
The {\tt service} function timeout value
prevents the R session from spinning and consuming lots of CPU time. See the
{\tt service} help page for more information.

\subsection{Sending data to clients}
The {\tt websocket\_write} and {\tt websocket\_broadcast} functions are
used to send data to connected clients. The {\tt websocket\_broadcast} function
emulates a true broadcast by sending data in a loop to all connected websocket
clients associated with the specified server. 

The {\tt websocket\_write} function may be used at any time to send data
to a specific websocket client. Each websocket server environment returned
by the {\tt create\_server} function maintains a list of connected client
sockets in the variable {\tt client\_sockets}. Each client socket is in
turn represented by an R list. The following example assumes that the
{\tt server} environment has been initialized and contains at least one
connected client:
\begin{lstlisting}
websocket_write("Hello", server$client_sockets[[1]])
\end{lstlisting}
{\bf Note the use of the double bracket indexing operator to select a single
list element from the {\tt client\_sockets} list.}

\subsection{Close the server when done}
Servers should be closed when done as follows:
\begin{lstlisting}
websocket_close(server)
\end{lstlisting}

\subsection{HTTP convenience functions}

The {\tt websockets} package includes two convenience function closures for
servicing basic HTTP requests to non-websocket clients: {\tt
static\_file\_service} and {\tt static\_text\_service}. The functions take
either a file name or text string that contains an HTML web page, respectively,
and issue a well-formed HTTP 200 response to the requesting client. They are
intended to be used in the {\tt webpage} argument to the {\tt create\_server}
function.  These functions may be used to furnish web browser clients with
an HTML page that contains Javascript code to establish a websocket connection
to R.

The following example defines a basic web page:
\newpage
\begin{lstlisting}
content='<html><body>
  <script>
  socket = new WebSocket("ws://localhost:7681", "chat");
  try {
    socket.onmessage = function got_packet(msg) {
      document.getElementById("output").textContent = msg.data;
    } 
  catch(ex) {document.getElementById("output").textContent = "Error: " + ex;}
  </script>
  <div id="output"> SOCKET DATA APPEARS HERE </div>
  </body></html>
'

server = create_server(webpage=static_text_service(content))
\end{lstlisting}
The web page text will be issued to any client making an HTTP GET request.  To
serve content from files instead, use the {\tt static\_file\_service} function.
Additionally, the {\tt static\_file\_service} function checks to see if the
file has been updated and always uses a fresh version.

Note that both convenience functions mostly ignore the GET RESOURCE and all
other GET or POST request parameters. They always only return the specified
HTML content. See the section on using the package as a generic HTTP service
below for more comprehensive examples.


\section{R as a websocket client}

The {\tt websockets} package includes the {\tt websocket} function for creating
clients that can interact with other websocket services.  It supports protocol
versions 00 and newer protocols up to at least version 08 (version 00 and 08
are most widely used by web browsers at the time of this writing).

The {\tt websocket} function returns an environment similar to the {\tt
create\_server} function, with a single list element in the enclosed {\tt
client\_sockets} variable corresponding to the client. Set callback functions
on the new client context to handle websocket events just as outlined above for
websocket servers. And use the {\tt websocket\_write} function exactly as
outlined about to write data through the client connection to the connected
server.

The following example connects to a publicly available websocket echo
server (using the 00 protocol).
\newpage
\begin{lstlisting}
> library(websockets)
> client = websocket("ws://echo.websocket.org", port=80)
> set_callback("receive", function(DATA,WS,HEADER) cat(rawToChar(DATA)), client)

> websocket_write("Testing, testing", client)
[1] 1

> service(client)
Testing, testing

>websocket_close(client)
\end{lstlisting}
See the {\tt websocket} man page for more information.

\section{Using the {\tt websockets} package as a basic web server}

The websockets package includes functions that may be used to define a basic
(non-websocket) web service. See the
\href{http://cran.r-project.org/web/packages/Rook/}{Rook} package for an
alternate comprehensive R web service.

The {\tt webpage} argument to the {\tt create\_server} function specifies a
generic HTTP callback function. The callback function must take two arguments,
{\tt socket} and {\tt header}, that represent a low-level client socket
connection and the full HTTP header. Users are free to define arbitrary HTTP
handler functions to respond to the incoming generic HTTP request. The
{\tt websockets} package provides two helper functions for use in HTTP callbacks:
\begin{itemize}
\item {\tt {\bf http\_vars(socket, header)}}\\
Parse the HTTP header for GET or POST variables, returning them in a list.
\item {\tt {\bf http\_response(socket, status, content\_type, content)}}\\
Write a well-formed HTTP response back to the client socket, closing the connection
when done.
\end{itemize}

The following example illustrates a basic HTTP-only service that asks the user for a
stock ticker symbol and produces a historic plot of prices looked up from Yahoo Finance
using the
\href{http://cran.r-project.org/web/packages/quantmod/}{quantmod} package.
\newpage
\begin{lstlisting}
library("websockets")
library("caTools")
library("quantmod")

httpd = function(socket, header) {
  body = "<html><body><form>Ticker: <input type='text' name='symbol'/></form>"
  vars = http_vars(socket, header)
  if(!is.null(vars)) {
    getSymbols(vars$symbol)
    f = tempfile()
    jpeg(file=f, quality=100, width=650, heigh=450)
    chartSeries(get(vars$symbol), name=vars$symbol, TA=c(addVo(),addBBands()))
    dev.off()
    img = base64encode(readBin(f,what="raw",n=1e6))
    unlink(f)
    body = paste(body,"<br/><img src='data:image/jpg;base64,\n", img,"'</img>")
  }
  http_response(socket,content=charToRaw(paste(body,"</body></html")))
}

w = create_server(webpage=httpd, port=9999)
cat("Direct your browser to http://localhost:9999\n")
while(TRUE) service(w)
\end{lstlisting}

\section{Tips and miscellaneous notes}
\subsection{Binary data}
Binary data is supported by IETF websocket protocol versions greater than
00. The {\tt websockets} package supports the older 00 protocol with
ASCII-only data, as well as binary data transfers with newer clients.
At the date of this writing, the only commonly available web browser
supporting the new protocols is Google Chrome (browser version 14 and greater),
which uses the IETF version draft-ietf-hybi-thewebsocketprotocol-08.

JSON is probably a good non-binary choice to use when interacting with
Javascript and the data size is not too large. The suggested {\tt RJSONIO}
package helps map many native R objects to JSON and vice versa, greatly
facilitating interaction between R and Javascript. 

\subsection{Setting generic HTTP function handler callbacks}
Assign a generic HTTP handler with {\tt set\_callback} function using the
callback name ``static'' as in the following example:
\begin{lstlisting}
hello = function(socket, header) {
  http_response(socket,content=charToRaw("HELLO"))
}
server = create_server(port=9999)
set_callback('static', hello, server)
\end{lstlisting}
Alternatively, use the {\tt webpage} parameter in the {\tt create\_server}
function.

\subsection{Limitations}

The {\tt websockets} package does not yet automatically support the IETF
WebSocket message fragmentation protocol. Users may implement the protocol
manually, or limit the package use to single-frame messages.

The {\tt websockets} package uses the R options interface to set the
{\tt websockets\_max\_buffer\_size} variable to a default value of
16777216. Message frames exceeding this value will be truncated. Users may
change the maximum buffer size value by resetting the option.

Extensions as specified by the IETF draft specification are not yet supported.

High-level control of the data framing headers are not yet exposed to users,
but will be in a future package version.

\end{document}
